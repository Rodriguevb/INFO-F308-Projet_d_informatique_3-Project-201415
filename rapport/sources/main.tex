\documentclass[a4paper,10pt]{article}
\usepackage[a4paper]{geometry}
\geometry{hscale=0.8,vscale=0.8,centering}
\usepackage[utf8]{inputenc}
\usepackage[T1]{fontenc}
\usepackage[french]{babel} % Exposant


\usepackage{enumerate} % Listes
\usepackage{amsmath} % Matrices
\usepackage{graphicx}
\usepackage{amssymb}
\usepackage{ulem}
\usepackage{color}

\usepackage{listings} % Lecture du code
\usepackage{hyperref} % Hyperlien


% Mise en page spéciale fancyhdr pour les en-têtes
\usepackage{fancyhdr}
\pagestyle{fancy}

\renewcommand{\headrulewidth}{0pt}
\fancyhead[C]{} % Rien en haut de page au milieu
\fancyhead[L]{\leftmark} % Nom du chapitre actuel en haut de page à gauche
\fancyhead[R]{\thepage} % Numéro de page en haut de page à droite

\renewcommand{\footrulewidth}{0pt}
\fancyfoot[C]{} % Rien en bas de page au milieu
\fancyfoot[L]{Open source pour ajout ou modification: https://github.com/Rodriguevb/Resumes-et-formulaires-de-Universite-Libre-de-Bruxelles/} % Numéro de page en bas de page à gauche
\fancyfoot[R]{\thepage} % Numéro de page en bas de page à droite
\setlength{\headheight}{12.1638pt}


\author{Rodrigue \textsc{Van Brande}} % Auteur
\date{\today} % Date de compilation du pdf

\pdfinfo{
    /Author   (Magali Hublet & Nicolas HEREMAN & Rodrigue VAN BRANDE & Julien VANBERGEN)
    /Creator  (https://github.com/Rodriguevb/ProjetAnnee3)
}

%Création d'un subsub...section (avec \paragraph{} \subparagraph{})
\setcounter{secnumdepth}{5}
\setcounter{tocdepth}{5}

% Titre du PDF
\title{INFO-F-30x - Projet d'année\\Prénom \textsc{EtNomDuProf}\\Rapport}

\pdfinfo{
/Title(INFO-F-30x - Projet d'année - Prénom EtNomDuProf - Rapport)
}

% Profondeur pour la table des matières dans les titres
\setcounter{tocdepth}{3}

\begin{document}
    \maketitle       % Titre
    \newpage         % Saut de page
    \tableofcontents % Table des matières / Besoin d'une double compilation

     \paragraph{} Les test ont été effectués sur des matrices 7x7 dus aux limitations de contraintes dans la version étudiant de AMPL avec le solveur knitro (300 contraintes maximum).  De ce fait, nous nous sommes limités également au premier modèlé décrit dans l'article pour AMPL.

\paragraph{} Les résultats de la version mathématiques sont plus précis, mais plus lent (car la limite de contrainte nous empêche de l'optimiser). De plus, le temps augmente expentionellement avec le nombre de lampadaires tandis que l'énumération exhaustive montre une progression linéaire. Cela peut être observé sur les graphiques ci-dessous.


\begin{figure}
        \begin{minipage}{.5\textwidth}
        \subfloat[AMPL]{\includegraphics[scale=0.25]{./image/graphiqueAMPL.png}}
        \end{minipage}
        \begin{minipage}{.5\textwidth}
        \subfloat[Backtracking]{\includegraphics[scale=0.25]{./image/graphiqueBacktrack.png}}
        \end{minipage}
\end{figure}


\paragraph{} Quant aux résultats, la puissance variable permet au mieux de se rapprocher de la demande. Ici, l'on peut voir les résultats pour les 3 algorithmes.

\begin{figure}
        \begin{minipage}{.5\textwidth}
        \subfloat[Exemple de coloriage]{\includegraphics[scale=0.25]{./image/plateauColoriage1.jpg}}\par
         \subfloat[Résultat AMPL]{\includegraphics[scale=0.25]{./image/ampl1.jpg}}
        \end{minipage}
        \begin{minipage}{.5\textwidth}
        \subfloat[Résultat Backtracking]{\includegraphics[scale=0.25]{./image/backtrack1.jpg}}\par
        \subfloat[Résultat Backtracking rapide]{\includegraphics[scale=0.25]{./image/backtrackFast1.jpg}}
        \end{minipage}
\end{figure}



\end{document}
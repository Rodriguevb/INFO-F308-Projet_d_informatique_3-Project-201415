\section{Un \'eclairage public "intelligent"}



Un \'eclairage est dit "intelligent" lorsqu'il s'adapte au lieu et de l'heure ainsi qu'\`a la m\'et\'eo. Ainsi une nuit avec ciel d\'egag\'e d'\'et\'e ne demanderait que tr\`es peu de lumi\`ere qu'une nuit d'hiver une brume \'epaisse.



\section{Un \'eclairage public "\'economique"}



Un \'eclairage est dit "\'economique" lorsque son co\^ut d'utilisation est faible mais aussi si son co\^ut \`a l'installation et \`a la maintenance (remplacement d'ampoule, c\^ables, ...) le sont aussi.

Effectivement, un \'eclairage o\`u son co\^ut d'utilisation est relativement faible est bien mais si son installation est hors budget, elle ne verra pas le jour et perd tout int\'er\^et.

 L'inverse est vraie aussi ; une installation faible ne garantit en aucun cas d'avoir le budget pour l'utiliser.

Ainsi la somme de ces trois co\^uts permettraient d'\'evaluer la pertinence d'un \'eclairage \'a un autre.



\section{Un \'eclairage public "\'ecologique"}



Un \'eclairage est dit "\'ecologique" lorsqu'il ne laisse aucun ou tr\`es peu de trace \`a l'environnement. Le choix du mat\'eriel  recyclable est certes important, mais il faut aussi penser \`a d'autres facteurs comme le d\'egagement de $CO_2$ ou de mercure.



\section{Un \'eclairage public "efficace"}



Un \'eclairage est dit "efficace" lorsqu'il limite la pollution lumineuse si un \'eclairage intensif n'est pas n\'ecessaire. Par exemple mettre un lampadaire devant une fen\^etre d'une maison est \`a \'eviter.




\section{Un \'eclairage public parfait}


Le projet consiste \`a relier tous ces caract\'eristiques facilement pour concevoir le meilleur \'eclairage public d'un lieu donn\'e.
Pour r�soudre ce probl�me, nous allons utilis� un algorithme lin�aire comme celui pr�sent� dans l'article du European Journal Operational Research \cite{mainref}. 
L'espace sera organis� sous forme de grille afin de facilit� les calculs.


Nous tenterons de d�velopper un programme C++ facile d'utilisation qui permettra de fournir un placement optimis� des sources de lumi�re d'une route en fonction des donn�es re�ues.
L'algorithme se basera sur une grille construite � partir d'un r�seau routier quelconque et placera de fa�on optimale les sources de lumi�re.

Valeurs prises en compte par l'algorithme :
- Type de route ;
- Fr�quentation de la route ;
- Vitesse maximale autoris�e ;
- Co�fficient de dangerosit� ;
- M�t�o ;
- Lumi�re du soleil (heure) ;
- Budget � disposition.

R�sultat de l'algorithme :
- Mat�riel utilis� (�conomique et �cologique) ;
- Intensit� de la lampe ;
- Distance entre les sources des lumi�res ;
- Hauteur des sources des lumi�res ;
- Co�t d'installation ;
- Co�t d'utilisation (graphe en fonction de l'heure) ;
- Co�t de maintenance.
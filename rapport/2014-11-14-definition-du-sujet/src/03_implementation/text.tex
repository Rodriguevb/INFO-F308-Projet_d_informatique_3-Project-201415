Pour r\'esoudre ce probl\`eme, nous allons utilis\'e un algorithme lin\'eaire comme celui pr\'esent\'e dans l'article du European Journal Operational Research \cite{mainref}. Nous tenterons de d\'evelopper un programme C++ facile d'utilisation qui permettra de fournir un placement optimis\'e des sources de lumi\`ere d'une route en fonction des donn\'ees re\c ues. L'algorithme se basera sur une grille construite \`a partir d'un r\'eseau routier quelconque et placera de fa\c on optimale les sources de lumi\`ere.

\subparagraph{Valeurs prises en compte par l'algorithme :}
\begin{description}
	\item [Type de route], une autoroute demandera moins de lumière qu'une route au centre d'une ville ;
    \item [Fr\'equentation de la route], car un sentier au milieu d'un champ est peu pertinant ;
    \item [Vitesse maximale autoris\'ee], car le temps de r\'eaction est clairement influenc\'e par la visibilit\'e ;
    \item [Type de route] ;
    \item [Fr\'equentation de la route] ;
    \item [Vitesse maximale autoris\'ee] ;
    \item [Co\'efficient de dangerosit\'e] ;
    \item [M\'et\'eo] ;
    \item [Lumi\`ere du soleil] (heure) ;
    \item [Budget \`a disposition].
\end{description}

\subparagraph{R\'esultat de l'algorithme :}
\begin{description}
    \item [Mat\'eriel utilis\'e], \'economique mais aussi \'ecologique ;
    \item [Intensit\'e de la lampe],  ;
    \item [Distance entre les sources des lumi\`eres] ;
    \item [Hauteur des sources des lumi\`eres] ;
    \item [Co\^ut d'installation] ;
    \item [Co\^ut d'utilisation] (graphe en fonction de l'heure) ;
    \item [Co\^ut de maintenance].
\end{description}

% Présentation

L'\'eclairage public est l'ensemble des moyens mis en place pour \'eclairer les espaces publics. C'est un domaine qui ne doit pas \^etre sous-estim\'e car il a de grandes cons\'equences sur notre vie \`a tous, que ce soit au niveau de sa qualit\'e ou de la s\'ecurit\'e. \cite{projwebsite}

% Expliquer que c'est très très couteux.

Malheureusement cela a un certain co\^ut financier et \'energ\'etique et avec les risques de p\'enurie d'\'electricit\'e qui augmentent, il est important de faire des \'economies. En moyenne en Belgique, ce co\^ut est \'elev\'e \`a 53\% de la consommation \'electrique \`a la charge d'une commune. Selon l'Ademe \cite{ademe}, l'\'eclairage actuel pourrait \^etre tr\`es co\^eteux pour le financement public. Ainsi une am\'elioration de l'efficacit\'e \'energ\'etique pourrait r\'eduire la facture de moiti\'e.

% Heureusement il existe des solutions pour r\'eduire la facture.

Depuis plusieurs ann\'ee, les pouvoirs publics exp\'erimentent des extinctions \`a certaines heures de la nuits pour r\'eduire la facture. L'\'eclairage \'etant important pour la s\'ecurit\'e, il faut trouver d'autres moyen d'\'economie.
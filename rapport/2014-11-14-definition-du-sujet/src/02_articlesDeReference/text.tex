
%Article de r�f�rence du site du projet
\paragraph{}
Le premier article de r\'ef\'erence \cite{mainref} parle de programmation lin\'eaire pour r\'esoudre les problèmes de localisation optimale avec un syst\`eme bas\'e sur des grilles. Il commence par un bref historique des recherches \`a ce sujet, pour ensuite poser  un probl\`eme de ce style et montrer une fa\c{c}on de le r\'esoudre.



% Th�se de doctorat
\paragraph{}
La deuxi\`eme r\'ef\'erence \cite{phdref} est une th\`ese de doctorat sur le m\^eme sujet que l'article pr\'ec\'edent. Il parle donc de la m\^eme m\'ethode de r\'esolution en allant plus dans les d\'etails. Cela permet d'am\'eliorer la connaissance du sujet et donc, de faciliter le d\'eveloppement du projet.

% Recommandations techniques pour l eclairage public
\paragraph{}
La troisi\`eme article de référence \cite{ascenref} est une brochure sur les détails techniques de l'éclairage public (type de lampe, portée, distances, etc...). Cela pourrait nous \^etre très utile afin de savoir quel type de lampe est le plus approprié selon l'endroit à éclairer.

% Eclairage tunnel
\paragraph{}
La quatri\`eme article de référence \cite{tunnelref} est un article sur l'optimisation de l'éclairage dans les tunnels via une méthode non-linéaire. Cette méthode permet d'économiser de 7 à 50\% d'énergie dans l'éclairage des tunnels.

% Eclairage LED
\paragraph{}
La cinqui\`eme article de référence \cite{ledref} est un article concernant l'optimisation de l'éclairage LED sur les axes routiers via une lentille de forme libre. Ce type de lampadaire prend en compte le type de revêtement de la route mais permet également d'optimiser l'énergie ainsi que les effets d'éblouissement.

% Maximal Covering Location Problem
\paragraph{}
Notre sixième article de référence \cite{maxcovref} est un article assez général concernant le fait que l'on désire une couverture optimale, tout en disposant d'un nombre limité de sources. Il nous expose diverses solutions possibles ainsi que leur performance respective. Dans le cadre de ce projet, nous sommes tout particulièrement intéressé par le paragraphe nous indiquant une manière de résoudre ce problème à l'aide de la programmation linéaire.

% Optimizing scale and spatial Location
\paragraph{}
Ce septième article \cite{logisticref} adresse un cas plus concret en ce qui concerne l'optimisation en termes de tailles ainsi que d'emplacement de différents centres logistiques. En premier lieu, nous avons une introduction théorique, mais ensuite, nous avons un cas pratique venant vérifier ce qui a été dit plus haut. Un cas concret est un excellent moyen d'assimiler des concepts fort abstraits et de nous aider à mieux comprendre ce qui doit être fait.

% On the use of genetic algorithms
\paragraph{}
Cet article \cite{genref} nous parle des algorithmes génétiques (genetic algorithm) et nous les présente comme une procédure alternative à la résolution des problèmes d'optimisation de location. Il nous explique les différentes performances de ces algorithmes dans certaines conditions bien spécifiques. Grâce à cela, ill identifie les cas où de tels algorithmes peuvent être une alternative utile, voir même supérieure en comparaison avec les méthodes traditionnelles.

%Article de r�f�rence du site du projet
\paragraph{}
Le premier article de r\'ef\'erence \cite{mainref} parle de programmation lin\'eaire pour r\'esoudre ce genre de probl\`eme avec un syst\`eme bas\'e sur des grilles. Il commence par un bref historique des recherches \`a ce sujet. Ensuite, il pose un probl\`eme de ce style et montre une fa\c{c}on de le r\'esoudre.



% Th�se de doctorat
\paragraph{}
La deuxi\`eme r\'ef\'erence \cite{phdref} est une th\`ese de doctorat sur le m\^eme sujet que l'article pr\'ec\'edent. Il parle donc de la m\^eme m\'ethode de r\'esolution en allant plus dans les d\'etails. Cela permet d'am\'eliorer la connaissance du sujet et donc de faciliter le d\'eveloppement du projet.

% Recommandations techniques pour l eclairage public
\paragraph{}
La troisi\`eme article de reference \cite{ascenref} est une brochure sur les détails techniques de l'éclairage public (type de lampe, portée, distances, etc...). Cela est très utile pour savoir quel type de lampe est approprié selon l'endroit à éclairer.

% Eclairage tunnel
\paragraph{}
La quatri\`eme article de reference \cite{tunnelref} est un article sur l'optimisation de l'éclairage dans les tunnels via une méthode non-linéaire. Cette méthode permet d'économiser de 7 à 50% d'énergie dans l'éclairage des tunnels.

% Eclairage LED
\paragraph{}
La cinqui\`eme article de reference \cite{ledref} est un article sur l'optimisation de l'éclairage LED sur les axes routiers via une lentille de forme libre. Ce type de lampadaire prend en compte le type de revêtement de la route, optimise l'énergie et les effets d'éblouissement.
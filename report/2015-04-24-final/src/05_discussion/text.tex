 \paragraph{}
Les résultats obtenus dans la partie AMPL sont les meilleurs car la puissance des lampadaires est variable et on s'approche très près de la matrice de demande. Ces résultats sont dus aux équations décrites dans l'article scientifique et surtout à la puissance variable des lampadaires. Cette variabilité des puissances permet de couvrir au mieux la zone et de s'adapter à ces besoins. 

\paragraph{}
Dans les algorithmes d'énumération, la puissance fixe permet de trouver une solution plus rapidement car on ne doit pas chercher quelle puissance est la meilleure. Pour des tailles de tableaux plus grands, l'énumération devient vite impossible car il y a beaucoup trop de cas à prendre en compte\footnote{Total = largeur*hauteur*nombre de lampadaire}. C'est pour cela qu'un algorithme d'énumération s'arrêtant à un pourcentage donné de satisfaction a été donné. Cela permet d'avoir une résultat correct sans prendre trop de temps.

\paragraph{}
La partie AMPL peut être amélioré avec les modèles 2 et 3 de l'article de référence afin de diminuer significativement son temps de calcul. Cela serait possible si il n'y avait pas une limite de 300 contraintes dans la version étudiante du solveur KNITRO car il faut encore rajouter de nouvelles contraintes.



Afin de pouvoir trouver des m\'{e}thodes pour r\'{e}soudre le probl\`{e}me, il est important de le d\'{e}finir correctement. Ce que nous voulons, c'est placer des lampadaires afin d'obtenir une luminosit\'{e} le plus proche possible de ce qui est demand\'{e}.

\section{Mod\'{e}lisation du probl\`{e}me}
Pour pouvoir r\'{e}soudre le probl\`{e}me, nous devons d'abord le mod\'{e}liser.  Nous divisons l'endroit \`{a} \'{e}clairer sous forme de matrice dont nous remplissons chaque case d'un nombre repr\'{e}sentant la luminosit\'{e} voulue \`{a} cet emplacement.

La taille de cette matrice est en lien direct avec la pr\'{e}cision des r\'{e}sultats et du temps de calcul. Lorsque le nombre de cases augmente, leur taille diminue. Chaque case repr\'{e}sente donc une partie plus pr\'{e}cise de l'endroit \`{a} \'{e}clairer. Le r\'{e}sultat sera donc lui aussi plus pr\'{e}cis. Mais si le nombre de case augmente, le nombre de solution possible et donc de calcul \`{a} effectuer augmente aussi. Le temps d'ex\'{e}cution sera donc plus long.
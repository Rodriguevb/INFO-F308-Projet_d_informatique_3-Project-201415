 \paragraph{} En conclusion, nous avons un logiciel qui permet de trouver l'emplacement optimal des lampadaires. D'une part, via un modèle mathématique qui prend en compte des puissances variables; et d'autre part, via une énumération exhaustive qui a une puissance fixe mais des zones bloquantes.

 \paragraph{} Dans le futur, avec plus de contraintes et certaines optimisations pour gagner du temps de calcul, on pourrait utiliser ce genre de logiciel pour éclairer les routes, parc, villes, etc\ldots afin de permettre aux gens de circuler en sécurité dans les lieux publics. Pour cela, il suffirait d'ajouter des contraintes pour bloquer des zones, prendre en compte les heures les plus fréquentées. 

 \paragraph{} De plus, on voit déjà actuellement un essor pour les lampadaires intelligent\footnote{La société CityBox, filiale de Bouygues Construction}. Cela permetterait de réduire encore les couts de l'éclairage tout en augmentant la sécurité des usagers. C'est probablement ce qui nous attend dans le futur.
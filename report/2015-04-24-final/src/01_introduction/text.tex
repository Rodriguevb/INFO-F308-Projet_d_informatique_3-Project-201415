 
% Introduction
 
L'\'eclairage public est un terme g\'en\'eral repr\'esentant l'ensemble des moyens mis en place afin d'illuminer les espaces publics. C'est un domaine qui ne doit pas \^etre sous-estim\'e car il a de grandes cons\'equences sur notre vie \`a tous, que ce soit au niveau de sa qualit\'e ou de la s\'ecurit\'e. \cite{projwebsite}

 
 
% Expliquer que c'est tr\`es tr\`es couteux.

Malheureusement, cela a un certain co\^ut, aussi bien financier qu'\'energ\'etique, et  les risques de p\'enurie d'\'electricit\'e ne font qu'augmenter. Tout cela nous indique qu'il devient urgent de r\'ealiser des \'economies dans ce domaine. En effet, en moyenne, en Belgique, ce co\^ut  s'\'eleve \`a 53\% de la consommation \'electrique \`a la charge d'une commune. De plus, selon l'Ademe \cite{ademe}, l'\'eclairage actuel pourrait \^etre tr\`es co\^uteux pour le financement public. Ainsi, une am\'elioration de l'efficacit\'e \'energ\'etique pourrait r\'eduire la facture de moiti\'e.



% Heureusement il existe des solutions pour r\'eduire la facture.

Depuis plusieurs ann\'ees, les pouvoirs publics exp\'erimentent des extinctions d'\'eclairage \`a certaines heures de la nuit afin de r\'eduire ces co\^ut. Cependant, l'\'eclairage est un \'el\'ement essentiel \`a notre \ s\'ecurit\'e, c'est pourquoi nous nous devons de trouver d'autres moyens d'\'economie.



% L'optimisation du placement

Un \'eclairage poss\`ede un placement dit "optimis\'e" lorsqu'il s'adapte au contexte, c'est à dire au terrain dans lequel il se trouve. En effet, certains endroits demandent plus de lumi\`ere que d'autres. Par exemple, les chemins passants au bord de l'eau. Chaque ann\'ee, plusieurs personnes meurent noy\'ees car les endroits au bord de l'eau sont peu ou pas \'eclair\'es. Mais la pollution lumineuse a un impact tr\`es n\'egatif sur les animaux. Par exemple, les abeilles sont fortement attir\'ees par la lumi\`ere. Ces petites ouvri\`eres s'activent à la premi\`ere lueur aperçue. C'est l'une des nombreuses causes de leur disparition pr\`es des zones rurales car elles meurent de fatigue ! Il ne suffit pas de mettre des lampes partout ! Pour des soucis écologiques, on souhaiterait utiliser le moins de lampes possibles.



% En gros se qu'on fait

Le but de notre projet parait donc simple ; nous devons placer un nombre minimum de lampadaires dans un espace donné afin d'éclairer la zone de manière optimale.


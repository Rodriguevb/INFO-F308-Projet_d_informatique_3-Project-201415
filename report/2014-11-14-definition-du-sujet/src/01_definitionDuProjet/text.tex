
% Présentation

L'\'eclairage public est un terme g\'en\'eral repr\'esentant l'ensemble des moyens mis en place afin d'illuminer les espaces publics. C'est un domaine qui ne doit pas \^etre sous-estim\'e car il a de grandes cons\'equences sur notre vie \`a tous, que ce soit au niveau de sa qualit\'e ou de la s\'ecurit\'e. \cite{projwebsite}

% Expliquer que c'est très très couteux.

Malheureusement, cela a un certain co\^ut, aussi bien financier qu'\'energ\'etique, et  les risques de p\'enurie d'\'electricit\'e ne font qu'augmenter. Tout cela nous indique qu'il devient urgent de r\'ealiser des \'economies dans ce domaine. En effet, en moyenne, en Belgique, ce co\^ut  s'\'eleve \`a 53\% de la consommation \'electrique \`a la charge d'une commune. De plus, selon l'Ademe \cite{ademe}, l'\'eclairage actuel pourrait \^etre tr\`es co\^uteux pour le financement public. Ainsi, une am\'elioration de l'efficacit\'e \'energ\'etique pourrait r\'eduire la facture de moiti\'e.

% Heureusement il existe des solutions pour r\'eduire la facture.

Depuis plusieurs ann\'ees, les pouvoirs publics exp\'erimentent des extinctions d'\'eclairage \`a certaines heures de la nuit afin de r\'eduire ces co\^ut. Cependant, l'\'eclairage est un élément essentiel \`a notre \ s\'ecurit\'e, c'est pourquoi nous nous devons de trouver d'autres moyens d'\'economie.

Un \'eclairage est dit "intelligent" lorsqu'il s'adapte au contexte, c'est à dire, au lieu, à l'heure ainsi qu'\`a la m\'et\'eo. En effet, une nuit avec un ciel d\'egag\'e d'\'et\'e ne demande que tr\`es peu de lumi\`ere comparé à une nuit d'hiver avec une brume \'epaisse. Il ne faudrait donc pas éclairer de la même manière dans ces deux situations, or, actuellement, aucune distinction n'est faite. 

Un \'eclairage est dit "\'economique" lorsque son co\^ut d'utilisation est faible mais également lorsque son co\^ut d'installation et de maintenance (remplacement d'ampoules, de c\^ables, ...) le sont aussi.

Effectivement, un système d'\'eclairage disposant d'un co\^ut d'utilisation relativement faible est attractif mais si son installation est hors de prix, ce système perd tout son intérêt et ne verra jamais le jour. La proposition inverse est également vraie. En effet, ce n'est pas parce qu'un système d'éclairage dispose d'un coût faible d'installation qu'on aura le budget pour le mettre en place si, lors de son utilisation, il demande trop de ressources.

En conclusion, la somme de ces trois co\^uts nous permettraient d'\'evaluer la pertinence d'un certain type d'\'eclairage en comparaison avec un autre.

Un \'eclairage est dit "\'ecologique" lorsqu'il ne laisse aucune, ou tr\`es peu de traces sur l'environnement. Le choix de mat\'eriaux  recyclables est certes important, mais il faut aussi penser \`a d'autres facteurs tels que le d\'egagement de $CO_2$ ou de mercure.

Un \'eclairage est dit "efficace" lorsqu'il limite la pollution lumineuse quand cela n'est pas n\'ecessaire. Par exemple, mettre un lampadaire devant la fen\^etre d'une maison est \`a \'eviter, à moins que cela ne soit indispensable.

Ce projet consiste \`a relier toutes ces caract\'eristiques, de manière facile et optimale, dans l'optique de concevoir le meilleur \'eclairage public pour un lieu donn\'e.